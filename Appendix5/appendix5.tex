%!TEX root = ../thesis.tex
% ******************************* Thesis Appendix B ****************************

\chapter{Accuracy needed for comparators offset measurement}
\label{app:acc-offset}

Let us assume, for the average voltage of the offset N samples ($x_i ~ N(\overline{x}^*, \sigma^2)$) The average and the standard deviation is thus given by:
\begin{equation}
    \overline{x} = \frac{1}{N}\sum_{i=0}^{N-1}{x_i}
\end{equation}
and
\begin{equation}
    \sigma_{offset} = \sqrt{\frac{1}{N-1}\sum_{i=0}^{N-1}{(x_i - \overline{x})^2}}
\end{equation}

The error committed with respect to the true offset value ($\overline{x}^*$) is calculated by:
\begin{equation}
    \varepsilon_{\overline{x}} = \overline{x}^*-\overline{x} = \overline{x}^* - \frac{1}{N}\sum_{i=0}^{N-1}{x_i}
\end{equation}

Thus a non-biased measure gives an expectation of the error tending to zero. This is equivalent to say that
the probes have no offset and have been calibrated.

In consequence, the error for a given measure comes from the standard deviation of a measure: the
precision of probes.

\begin{equation}
    Var\left(\varepsilon_{\overline{x}}\right) = Var\left(\overline{x}^*\right) + Var\left( \right) = \frac{1}{N^2} \sum_{i=0}^{N-1}{\sum_{j\neq i}{(x_i-x_j)^2}}
\end{equation}

Since each sample have the same variance and supposed sufficiently separated in the time to not be
correlated (memory effect of the probe), the variance of the error can be simplified into

\begin{equation}
    Var\left(\varepsilon_{\overline{x}}\right) = \frac{\sigma^2}{N}
\end{equation}

So to limit the error on the offset to 5\% of the expected offset’s standard deviation for 10 samples, the $\sigma$
should be less than

\begin{equation}
    \sigma < \sigma_{offset} \sqrt{10}\frac{5}{100} = 3 mV \times 0.158 = 474 \mu V
\end{equation}